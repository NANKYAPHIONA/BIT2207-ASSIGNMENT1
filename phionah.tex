\documentclass[14pt]{article}
\usepackage{graphicx}
\begin{document}

\begin{center}\begin{Huge} MAKERERE UNIVERSITY\end{Huge}\end{center}

\begin{Huge} COLLEGE OF COMPUTING AND INFORMATION SCIENCES \end{Huge}


\begin{flushleft}\begin{huge} NAME: NANKYA PHIONA \end{huge}\end{flushleft}
\begin{flushleft}\begin{huge} REG.NO: 15/U/21729 \end{huge}\end{flushleft}
\begin{flushleft}\begin{huge} STUD.NO: 215023051 \end{huge}\end{flushleft}

\title{A REPORT ON HOW TO PREPARE PLAIN RICE AND FRENCH BEANS}

\author{NANKYA PHIONA}

\date{\today}

\maketitle

\tableofcontents

\section{Abstract}

This report gives the different steps taken when preparing plain rice and french beans. As the reader will read, he/she will get to know what it takes to do it.

\section{Introduction}

I love the meal of plain rice and French beans so much; it can be tasty and so delicious that every person would feel like tasting.
How I prepare it at home:
First I put charcoal on a charcoal stove, get a polythene bag and a match box. Then I light the stove. Because we are five at home, I have to prepare one kilogram of rice. I wash my saucepan with water, soap and a utensil cleaner while the stove is catching up with fire.

\section{How To prepare rice}

Get carrots that are kept in the fridge, chop them into small pieces in to the clean saucepan, then get green papers from the fridge also, chop them into small pieces in to the same saucepan. After I measure four cups of water putting it in the same saucepan because one kilogram can consume four cups of water. I add salt to the mixture in the saucepan with small amount of cooking oil. Then I get the stove that is covered with fire and put on my saucepan covered with a saucepan covered. I wait until water gets boils and add in rice.

\section{How to prepare french beans}

I get the French beans from the fridge, start sorting them in batches and chopping them into small pieces. While doing that I be checking my rice, when there is a lot of fire I reduce on the charcoal since rice does need a lot of fire. Then I go back to my French beans to cut them. After I start chopping onions, tomatoes, carrots, and green papers to be put in my sauce.
When I observe that rice is ready, I put it on the kitchen table. Add more charcoal on the stove because French beans need at least much fire for it to be ready, get a saucepan that was washed, put it on the charcoal stove and wait for it until it dries up.
When it gets dry, I put cooking oil in it and wait till it gets ready for five minutes. Add onions in the saucepan, wait for some three-four minutes, add French beans, I fry them for at least fifteen minutes, add tomatoes, carrots, green paper and salt at once, fry again for  five minutes. Then I get curly powder but I mostly prefer “Simba mbili” add to the sauce and fry for two minutes, I then get royco, put in a small dish and mix it with some little amount of water, I stir and put the mixture in my sauce, I wait for two minutes and then remove it from the stove.

\section{Conclusion}
Start serving people with the delicious meal, so tasty and looks good.
Everyone should at least try this fantastic meal, because I can’t even tell how it tastes but a person who has ever tasted it knows how it tastes. Because me too I had never tasted it till my dearest friend told me about that secret but the day I first prepared the whole family were like “Ooh what a delicious meal”. 








\end{document}